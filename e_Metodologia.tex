%!TEX root = Protocolo.tex
%%%%%%%%%%%%%%%%%%%%%%%%%%%%%%%%%%%%%%%%%%%%%%%%%%%%%%%%%%%%
%% Capítulo 4 / Metodología
%%%%%%%%%%%%%%%%%%%%%%%%%%%%%%%%%%%%%%%%%%%%%%%%%%%%%%%%%%%%
\newpage
\setcounter{secnumdepth}{0}
\section{Metodología}
El diseño de la presente investigación se enfoca en explorar el reconocimiento de sonido y su importancia en la investigación, utilizando un enfoque basado en redes neuronales para su implementación en un prototipo embebido. El objetivo es comprender cómo el uso de estas redes puede mejorar la capacidad de reconocer y analizar patrones de sonido en diferentes contextos, y cómo esto puede tener aplicaciones relevantes en diversos campos de estudio. A continuación, se describe el diseño de la investigación para abordar este tema específico:

\subsection{Tipo de investigación}
Esta investigación se enmarca principalmente en exploratoria y descriptiva. Se busca explorar y describir el campo del reconocimiento de sonido y su relación con las redes neuronales, así como su importancia en la investigación en general. Además, se puede incluir un componente experimental para evaluar el rendimiento y la efectividad de las redes neuronales en el reconocimiento de sonido en un entorno controlado.

\subsection{Unidades de análisis}
Unidades de análisis: En esta investigación pueden ser diversos contextos donde el reconocimiento de sonido juega un papel relevante como la industria. Se pueden seleccionar casos específicos dentro de cada contexto para realizar un análisis detallado y obtener resultados más concretos.

\subsection{Técnicas de observación y recolección de datos}
Se utilizarán técnicas de observación directa, análisis de registros y entrevistas a expertos en el campo del reconocimiento de sonido y redes neuronales. Además, se pueden realizar pruebas o experimentos controlados para recopilar datos cuantitativos sobre el rendimiento de las redes neuronales en la identificación y análisis de diferentes tipos de sonido.

\subsection{Instrumentos y procedimientos}
Los instrumentos utilizados en esta investigación pueden incluir cuestionarios estructurados, hojas de registro de datos, sistemas de grabación de sonido y equipos de análisis de señales. Se diseñarán y adaptarán específicamente para capturar información relevante sobre los aspectos clave del reconocimiento de sonido y la aplicación de redes neuronales para la implementación en un prototipo embebido.

\subsection{Técnicas de análisis}
Los datos recopilados serán analizados utilizando técnicas cualitativas y cuantitativas, dependiendo de la naturaleza de los datos. Se realizarán análisis de contenido de entrevistas y observaciones, y se aplicarán técnicas estadísticas para analizar los resultados de las pruebas o experimentos.

El diseño de investigación propuesto permitirá explorar y describir de manera sistemática el reconocimiento de sonido y su importancia en la investigación, centrándose en el uso de redes neuronales como enfoque principal. Además, se busca obtener evidencia empírica sobre la efectividad y las aplicaciones de estas redes en el reconocimiento de patrones de sonido \cite{ues, martinez, osorio, frro2, apple}.
