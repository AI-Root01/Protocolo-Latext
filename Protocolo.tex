%%%%%%%%%%%%%%%%%%%%%%%%%%%%%%%%%%%%%
%%%%%%%%%%%% PREÁMBULO  %%%%%%%%%%%%%%%%%
%% Uso de la clase ITTol-Protocolo
%% IMPORTANTE: Compilar con LuaLaTeX
%%%%%%%%%%%%%%%%%%%%%%%%%%%%%%%%%%%
\documentclass[11pt]{ITTol-Protocolo}

% Paquete usado para incluir texto falso (Lorem Ipsum) en el ejemplo 
\usepackage{lipsum}					
\usepackage{pgfgantt}
\usepackage{tocloft}
% Para links sin decoración
%\hypersetup{colorlinks=false,allbordercolors=white}

\titulo{Aplicación de Redes Neuronales en el Reconocimiento de Sonido para la Investigación de Fallas en Ventiladores: Desarrollo de un Prototipo Embebido}
\autor{Frank Prime }
\grado{Maestría en Ciencias de la Ingeniería}
\date{Septiembre de 2023}
\director{Dr. R}
\codirector{Dr. F}                                      % Si existe un codirector

\newcommand{\todo}{\noindent \emph{\textcolor{red}{\textbf{ToDo~:}}} }

%%%%%%%%%%%%%%%%%%%%%%%%%%%%%%%%%%%%%
%%%%%%%%%%%% DOCUMENTO  %%%%%%%%%%%%%%%%%
%%%%%%%%%%%%%%%%%%%%%%%%%%%%%%%%%%%%%
\begin{document}

\maketitle	                                                                            % Crea la carátula
\frontmatter													% Numeración. en romano 

%%%%%%%%%%%%%%%%%%%%%%%%%%%%%%%%%%%%%
%%%% Resumen en español
%%\begin{abstract}                                                                   
%%  \textcolor{gray}{Se presentará en forma breve los aspectos relevantes de su Tesis, 
%%  como objetivos, resultados y posibles aplicaciones.  No deberá exceder a una cuartilla.}\\
%%
%%  \lipsum[1]										%Texto Lorem-Ipsum
%%\end{abstract}
%%
%%%%%%%%%%%%%%%%%%%%%%%%%%%%%%%%%%%%%
%%%% Resumen en inglés
%%\selectlanguage{english}                                                     
%%\begin{abstract}
%%\textcolor{gray}{The relevant aspects of your thesis will be briefly presented, such as objectives,
%%results and possible applications. It should not exceed one page.}\\
%%
%%\lipsum[2]										%Texto Lorem-Ipsum
%%\end{abstract}


%%%%%%%%%%%%%%%%%%%%%%%%%%%%%%%%%%
% Índices
\renewcommand{\contentsname}{Índice} 
\tableofcontents
                                                               %  Índice de Contenido
%{\noskip\listoffigures}                                                            % Índice de Figuras
%{\noskip\listoftables}                                                             % Índice de Tablas

\mainmatter												% Numeración en arábigo

%%%%%%%%%%%%%%%%%%%%%%%%%%%%%%%%%%
% Capítulos
%!TEX root = Protocolo.tex
%%%%%%%%%%%%%%%%%%%%%%%%%%%%%%%%%%%%%%%%%%%%%%%%%%%%%%%%%%%%
%% Capítulo 1 / Introducción
%%%%%%%%%%%%%%%%%%%%%%%%%%%%%%%%%%%%%%%%%%%%%%%%%%%%%%%%%%%%

\newpage
\setcounter{secnumdepth}{0}
\addcontentsline{toc}{chapter}{Capítulo 1}
\section{Introducción}

El reconocimiento de sonido es un campo de investigación relevante en la actualidad, con aplicaciones significativas en la industria y el procesamiento de señales de audio. Las redes neuronales han demostrado ser una herramienta eficaz para abordar diversos desafíos en estos ámbitos, como la reducción de costos de producción, la optimización de cadenas de suministro, la detección de defectos y la mejora de la calidad del producto en la industria de la maquinaria y fábricas. Asimismo, en el procesamiento de señales de audio, las redes neuronales son cruciales para mejorar la calidad del sonido, la separación de fuentes, la detección y eliminación de ruido, y la personalización del contenido auditivo.

Sin embargo, a pesar de los avances en el uso de redes neuronales en el reconocimiento de sonido, todavía existen desafíos y áreas de investigación que requieren atención. Además, se requiere investigar y desarrollar enfoques más eficientes y precisos para el procesamiento de señales de audio, con el fin de lograr una reproducción de sonido de alta calidad y una experiencia auditiva mejorada en los entornos industriales.

Por lo tanto, el presente estudio tiene como objetivo abordar estos desafíos y explorar las oportunidades de investigación en el campo del reconocimiento de sonido basado en redes neuronales, con un enfoque particular en la industria. Se busca identificar las áreas clave donde las redes neuronales pueden aportar mejoras significativas en términos de eficiencia, calidad del producto y experiencia auditiva.

En resumen, el problema planteado se centra en cómo aprovechar de manera óptima las redes neuronales en el reconocimiento de sonido para maximizar la eficiencia y calidad en la industria, abordando desafíos específicos en el procesamiento de señales de audio y su aplicación en entornos industriales mediante un prototipo embebido \cite{agudoestudio, atriaredes, garciamendoza, gomezarmenta, openwebinars, seorl2014audiologia}.                                                      	% Introducción.tex
%!TEX root = Protocolo.tex
%%%%%%%%%%%%%%%%%%%%%%%%%%%%%%%%%%%%%%%%%%%%%%%%%%%%%%%%%%%%
%% Capítulo 2 / Fundamentos Teóricos y Problemática
%%%%%%%%%%%%%%%%%%%%%%%%%%%%%%%%%%%%%%%%%%%%%%%%%%%%%%%%%%%%
\newpage
\setcounter{secnumdepth}{0}
\section{Fundamentos Teóricos}

%\epigrafe{Everyone in this country should learn how to program a computer, because it teaches you how to think.}
%              {\textsc{Steve Jobs}}
              
El fundamento teórico de esta investigación se basa en los siguientes puntos clave:

1. Redes Neuronales: Se analizarán las redes neuronales como modelos computacionales inspirados en el funcionamiento del cerebro humano. Se abordarán sus componentes fundamentales, como las neuronas, las conexiones sinápticas y los algoritmos de aprendizaje. Se explorarán diferentes tipos de redes neuronales, como las redes neuronales convolucionales (CNN) y las redes neuronales recurrentes (RNN), y se discutirá su aplicación en el reconocimiento de sonido \cite{garciamurillo, frro,gtocoma,ehu}.

2. Reconocimiento de Sonido: Se examinarán los conceptos y métodos utilizados para el reconocimiento de sonido. Se analizarán técnicas de extracción de características de señales de audio, como el análisis espectral y el análisis temporal, así como algoritmos de clasificación y detección de patrones aplicados al reconocimiento de sonido. Se revisarán enfoques tradicionales, como el modelado oculto de Markov (HMM), y se enfatizará en el uso de redes neuronales para mejorar la precisión y eficiencia del reconocimiento de sonido \cite{tesisipn}.

3. Procesamiento de Señales de Audio: Se explorarán los fundamentos del procesamiento de señales de audio en el contexto del reconocimiento de sonido en la industria. Se abordarán técnicas de filtrado, separación de fuentes, cancelación de ruido y mejora de la calidad del sonido. Se discutirán algoritmos y enfoques específicos utilizados para resolver desafíos comunes en entornos industriales, como la presencia de ruido de fondo y la variabilidad de las condiciones de grabación \cite{cevallos}.

4. Los ventiladores son dispositivos que generan una corriente de aire para refrescar y ventilar el ambiente en el que se encuentran, y son importantes para mantener una temperatura adecuada y prolongar la vida útil de los componentes en las máquinas. En la industria, los ventiladores industriales son esenciales para mantener la calidad del aire y la temperatura en los procesos productivos, y pueden resistir condiciones operativas severas. 

5. Aplicaciones Industriales: Se analizará el papel de las redes neuronales en la industria, específicamente en la reducción de costos de producción, la optimización de procesos y la mejora de la calidad del producto. Se presentarán casos de estudio y ejemplos de implementaciones exitosas de redes neuronales en entornos industriales, destacando los beneficios obtenidos, como la eficiencia mejorada, la detección de defectos y la automatización de tareas de inspección \cite{garciamurillo, gtocoma,cevallos,ehu}.

6. Estado del Arte: Se realizará una revisión exhaustiva de la literatura científica y técnica más actualizada en el campo del reconocimiento de sonido basado en redes neuronales. Se identificarán las investigaciones y avances más relevantes, así como las tendencias emergentes y las áreas de investigación prometedoras. Se resumirán las principales contribuciones y se establecerá la base para el desarrollo de la investigación propuesta \cite{scribd}.
\newpage
\subsection{Problematica}

El reconocimiento de sonido es un campo de investigación relevante en la actualidad, con aplicaciones significativas en la industria y el procesamiento de señales de audio. Las redes neuronales han demostrado ser una herramienta eficaz para abordar diversos desafíos en estos ámbitos, como la reducción de costos de producción, la optimización de cadenas de suministro, la detección de defectos y la mejora de la calidad del producto en la industria de la maquinaria y fábricas. Asimismo, en el procesamiento de señales de audio, las redes neuronales son cruciales para mejorar la calidad del sonido, la separación de fuentes, la detección y eliminación de ruido, y la personalización del contenido auditivo.

Sin embargo, a pesar de los avances en el uso de redes neuronales en el reconocimiento de sonido, todavía existen desafíos y áreas de investigación que requieren atención. Además, se requiere investigar y desarrollar enfoques más eficientes y precisos para el procesamiento de señales de audio, con el fin de lograr una reproducción de sonido de alta calidad y una experiencia auditiva mejorada en los entornos industriales.

Por lo tanto, el presente estudio tiene como objetivo abordar estos desafíos y explorar las oportunidades de investigación en el campo del reconocimiento de sonido basado en redes neuronales, con un enfoque particular en la industria. Se busca identificar las áreas clave donde las redes neuronales pueden aportar mejoras significativas en términos de eficiencia, calidad del producto y experiencia auditiva.

En resumen, el problema planteado se centra en cómo aprovechar de manera óptima las redes neuronales en el reconocimiento de sonido para maximizar la eficiencia y calidad en la industria, abordando desafíos específicos en el procesamiento de señales de audio y su aplicación en entornos industriales mediante un prototipo embebido \cite{agudoestudio, atriaredes, garciamendoza, gomezarmenta, openwebinars, seorl2014audiologia}.
                     		                 		% Fundamento.tex
%!TEX root = Protocolo.tex
%%%%%%%%%%%%%%%%%%%%%%%%%%%%%%%%%%%%%%%%%%%%%%%%%%%%%%%%%%%%
%% Capítulo 2 / Fundamentos Teóricos y Problemática
%%%%%%%%%%%%%%%%%%%%%%%%%%%%%%%%%%%%%%%%%%%%%%%%%%%%%%%%%%%%
\newpage
\setcounter{secnumdepth}{0}
\section{Estado del arte}

%\epigrafe{Everyone in this country should learn how to program a computer, because it teaches you how to think.}
%              {\textsc{Steve Jobs}}
              
El reconocimiento de sonido es un campo de investigación relevante en la actualidad, con aplicaciones significativas tanto en la industria como en el procesamiento de señales de audio. Las redes neuronales desempeñan un papel importante en este campo, brindando soluciones efectivas y eficientes. En la industria de la maquinaria, las redes neuronales se utilizan para reducir costos de producción, optimizar cadenas de suministro, detectar defectos y mejorar la calidad del producto, lo que conduce a una mayor eficiencia y minimización de errores. Por otro lado, en el procesamiento de señales de audio, las redes neuronales son cruciales para mejorar la calidad del sonido, la separación de fuentes, la detección y eliminación de ruido, y la personalización del contenido auditivo, lo que contribuye directamente a la calidad de los productos y procesos en la industria. Este documento aborda la importancia y las aplicaciones de las redes neuronales en el reconocimiento de sonido implementando un prototipo embebido, proporcionando un panorama de los avances y las oportunidades de investigación en este campo \cite{agudoestudio, garciamendoza, gomezarmenta, leotronics, matich2018redes, atriaredes}.
                     		                 		% EstadoArte.tex
%%%!TEX root = Protocolo.tex
%%%%%%%%%%%%%%%%%%%%%%%%%%%%%%%%%%%%%%%%%%%%%%%%%%%%%%%%%%%%
%% Capítulo 5 / Resultados
%%%%%%%%%%%%%%%%%%%%%%%%%%%%%%%%%%%%%%%%%%%%%%%%%%%%%%%%%%%%
\chapter{Resultados y discusión}
\epigrafe{Everyone in this country should learn how to program a computer, because it teaches you how to think.}
              {\textsc{Steve Jobs}}


\textcolor{gray}{Explicar detallamente los logros obtenidos acorde a los objetivos planteados y que sean concordantes con la hipótesis propuesta. Discutir los hallazgos considerando referencias relacionadas con el 
tema de estudio.}

\section{Ejemplo Tabla}
\lipsum[1].  
Los resultados pueden verse en la Tabla \ref{tab:Tabla1}

\begin{table}[htbp]
\centering
\caption{Resultados obtenidos.}
\label{tab:pversust}
\begin{tabular}{ccc}
\toprule
\textbf{Parametro A } & \textbf{Resultado } \\
\midrule
A & 1 \\
B & 2 \\ 
C & 3 \\
D & 4 \\
\bottomrule
\end{tabular}
\label{tab:Tabla1}
\end{table}

\lipsum[2]

\section{Ejemplo Figura}
\lipsum[3-4]

\begin{figure}[htbp]
\centering
\includegraphics[width=0.50\textwidth]{e_Imagen01}
\caption{Duis eget orci sit amet orci dignissim rutrum.}
\label{fig:particion}
\end{figure}

\section{Tercera sección}
\lipsum[5-6]

%!TEX root = Protocolo.tex
%%%%%%%%%%%%%%%%%%%%%%%%%%%%%%%%%%%%%%%%%%%%%%%%%%%%%%%%%%%%
%% Capítulo 4 / Metodología
%%%%%%%%%%%%%%%%%%%%%%%%%%%%%%%%%%%%%%%%%%%%%%%%%%%%%%%%%%%%
\newpage
\setcounter{secnumdepth}{0}
\section{Metodología}
El diseño de la presente investigación se enfoca en explorar el reconocimiento de sonido y su importancia en la investigación, utilizando un enfoque basado en redes neuronales para su implementación en un prototipo embebido. El objetivo es comprender cómo el uso de estas redes puede mejorar la capacidad de reconocer y analizar patrones de sonido en diferentes contextos, y cómo esto puede tener aplicaciones relevantes en diversos campos de estudio. A continuación, se describe el diseño de la investigación para abordar este tema específico:

\subsection{Tipo de investigación}
Esta investigación se enmarca principalmente en exploratoria y descriptiva. Se busca explorar y describir el campo del reconocimiento de sonido y su relación con las redes neuronales, así como su importancia en la investigación en general. Además, se puede incluir un componente experimental para evaluar el rendimiento y la efectividad de las redes neuronales en el reconocimiento de sonido en un entorno controlado.

\subsection{Unidades de análisis}
Unidades de análisis: En esta investigación pueden ser diversos contextos donde el reconocimiento de sonido juega un papel relevante como la industria. Se pueden seleccionar casos específicos dentro de cada contexto para realizar un análisis detallado y obtener resultados más concretos.

\subsection{Técnicas de observación y recolección de datos}
Se utilizarán técnicas de observación directa, análisis de registros y entrevistas a expertos en el campo del reconocimiento de sonido y redes neuronales. Además, se pueden realizar pruebas o experimentos controlados para recopilar datos cuantitativos sobre el rendimiento de las redes neuronales en la identificación y análisis de diferentes tipos de sonido.

\subsection{Instrumentos y procedimientos}
Los instrumentos utilizados en esta investigación pueden incluir cuestionarios estructurados, hojas de registro de datos, sistemas de grabación de sonido y equipos de análisis de señales. Se diseñarán y adaptarán específicamente para capturar información relevante sobre los aspectos clave del reconocimiento de sonido y la aplicación de redes neuronales para la implementación en un prototipo embebido.

\subsection{Técnicas de análisis}
Los datos recopilados serán analizados utilizando técnicas cualitativas y cuantitativas, dependiendo de la naturaleza de los datos. Se realizarán análisis de contenido de entrevistas y observaciones, y se aplicarán técnicas estadísticas para analizar los resultados de las pruebas o experimentos.

El diseño de investigación propuesto permitirá explorar y describir de manera sistemática el reconocimiento de sonido y su importancia en la investigación, centrándose en el uso de redes neuronales como enfoque principal. Además, se busca obtener evidencia empírica sobre la efectividad y las aplicaciones de estas redes en el reconocimiento de patrones de sonido \cite{ues, martinez, osorio, frro2, apple}.
 
%!TEX root = Protocolo.tex
%%%%%%%%%%%%%%%%%%%%%%%%%%%%%%%%%%%%%%%%%%%%%%%%%%%%%%%%%%%%
%% Capítulo 2 / Fundamentos Teóricos y Problemática
%%%%%%%%%%%%%%%%%%%%%%%%%%%%%%%%%%%%%%%%%%%%%%%%%%%%%%%%%%%%

\newpage
\setcounter{secnumdepth}{0}
\section{Cronograma de Actividades}

\begin{figure}[h]
	\centering
	
	\resizebox{\linewidth}{!}{%
		\begin{ganttchart}[
			y unit title=0.8cm,
			y unit chart=0.6cm,
			vgrid,
			hgrid,
			title label anchor/.style={below=-1.6ex},
			title left shift=.05,
			title right shift=-.05,
			title height=1,
			progress label text={},
			bar height=0.7,
			group right shift=0,
			group top shift=.6,
			group height=.3
			]{1}{16}
			%labels
			\gantttitle{Semestres}{16} \\
			\gantttitle{1}{4}
			\gantttitle{2}{4}
			\gantttitle{3}{4}
			\gantttitle{4}{4} \\
			%tasks
			\ganttbar[progress=100]{1. Revisión Bibliográfica}{1}{1} \\
			\ganttbar[progress=0]{2. Estado del Arte}{2}{8} \\
			\ganttbar[progress=0]{3. Protocolo}{2}{3} \\
			\ganttbar[progress=0]{4. Diseño de Experimento}{2}{5} \\
			\ganttbar[progress=0]{5. Colecciones de Datos}{3}{10} \\
			\ganttbar[progress=0]{6. Programación de Algoritmos}{5}{10} \\
			\ganttbar[progress=0]{7. Análisis de Resultados}{5}{11} \\
			\ganttbar[progress=0]{8. Presentación en Congreso}{6}{11} \\
			\ganttbar[progress=0]{9. Redacción de Artículo}{2}{12} \\
			\ganttbar[progress=0]{10. Escritura de Tesis}{2}{16} \\
		\end{ganttchart}%
	}
	\caption{Plan de Trabajo}
\end{figure}


                                                   	% Resultados
%%%!TEX root = Protocolo.tex
%%%%%%%%%%%%%%%%%%%%%%%%%%%%%%%%%%%%%%%%%%%%%%%%%%%%%%%%%%%%
%% Capítulo 6 / Conclusiones y Perspectivas
%%%%%%%%%%%%%%%%%%%%%%%%%%%%%%%%%%%%%%%%%%%%%%%%%%%%%%%%%%%%
\chapter{Conclusiones y recomendaciones}
\epigrafe{Everyone in this country should learn how to program a computer, because it teaches you how to think.}
              {\textsc{Steve Jobs}}
              
\textcolor{gray}{Concluir únicamente con los resultados obtenidos de la investigación de acuerdo a los objetivos y a la verificación de la hipótesis. Es importante comentar que este apartado no se debe de considerar como un resumen de resultados. Mencionar las recomendaciones que se consideren pertinentes, de acuerdo con las experiencias obtenidas, dentro del desarrollo del trabajo de tesis. Evitar los párrafos largos y las frases con varias ideas ( 6 a 7 renglones)}              

\section{Conclusiones}
\lipsum[1-3]

\section{Perspectivas}
\lipsum[1]									% Conclusiones, Perspectivas

\backmatter														% Los capítulos ya no se numeran

%%%%%%%%%%%%%%%%%%%%%%%%%%%%%%%%%%%
 %Apéndices o Anexos (Opcionales)
%%\appendix	
%%%!TEX root = Protocolo.tex
%%%%%%%%%%%%%%%%%%%%%%%%%%%%%%%%%%%%%%%%%%%%%%%%%%%%%%%%%%%%
%% Apéndice A / Demostraciones
%%%%%%%%%%%%%%%%%%%%%%%%%%%%%%%%%%%%%%%%%%%%%%%%%%%%%%%%%%%%
\chapter{Demostraciones}

\section{Explicando Teorema}
\lipsum[1-2].

\begin{proof} 
  Esta es mi demostración $a \times (b+c)=a \times b + a \times c$ 
\end{proof}
			        		% Demostraciones
%%%!TEX root = Protocolo.tex
%%%%%%%%%%%%%%%%%%%%%%%%%%%%%%%%%%%%%%%%%%%%%%%%%%%%%%%%%%%%
%% Apéndice B / Código Fuente
%%%%%%%%%%%%%%%%%%%%%%%%%%%%%%%%%%%%%%%%%%%%%%%%%%%%%%%%%%%%

%%%%%%%%%%%%%%%%%%%%%%%%%%%%%%%%%%%%%%%%%%%%%%%%%%%%% Preparación para el paquete listing
\definecolor{gray97}{gray}{.97}
\definecolor{gray75}{gray}{.75}
\definecolor{gray45}{gray}{.45}

\lstset{ frame=Ltb,
  framerule=0pt,
  aboveskip=0.5cm,
  framextopmargin=3pt,
  framexbottommargin=3pt,
  framexleftmargin=0.4cm,
  framesep=0pt,
  rulesep=.4pt,
  backgroundcolor=\color{gray97},
  rulesepcolor=\color{black},
  %
  stringstyle=\ttfamily,
  showstringspaces = false,
  basicstyle=\small\ttfamily,
  commentstyle=\color{gray45},
  keywordstyle=\bfseries,
  %
  numbers=left,
  numbersep=15pt,
  numberstyle=\tiny,
  numberfirstline = false,
  breaklines=true,
}
% Personalizando los ambientes para listing
\lstnewenvironment{listing}[1][]{
  \lstset{#1}\pagebreak[0]
}
{\pagebreak[0]}
\lstdefinestyle{terminal}{			% Terminal
  basicstyle=\scriptsize\bf\ttfamily,
  backgroundcolor=\color{gray75},
}
\lstdefinestyle{C} {				% Código en C
  language=C,
}

%%%%%%%%%%%%%%%%%%%%%%%%%%%%%%%%%%%%%%%%%%%%%%%%%%%%%%%%%%%%
\chapter{Código Fuente}

\section{Explicando un programa}
\lipsum[1-2]

\noindent
Escribe el siguiente programa en un archivo llamado \texttt{hola.c}:

\begin{lstlisting}[style=C]
#include <stdio.h>

int main(int argc, char* argv[]) 
{
    printf("Hola mundo!");
}
\end{lstlisting}

\noindent
Ahora compila usando \texttt{gcc}:

\begin{listing}[style=terminal, numbers=none]
$ gcc  -o hola hola.c
\end{listing}			          	  	 			% Diagramas
%\include{i_ApendiceC-Latex}			          	  	 			% Referencia rápida a LaTeX

%% Bibliografía
%%%%%%%%%%%%%%%%%%%%%%%%%%%%%%%%%%%
%\bibliographystyle{apacite}
%\renewcommand{\bibname}{Referencias}
\bibliographystyle{unsrt} 
\bibliography{biblio}											% Archivo .bib
\nocite{*}

\end{document}