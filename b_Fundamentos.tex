%!TEX root = Protocolo.tex
%%%%%%%%%%%%%%%%%%%%%%%%%%%%%%%%%%%%%%%%%%%%%%%%%%%%%%%%%%%%
%% Capítulo 2 / Fundamentos Teóricos y Problemática
%%%%%%%%%%%%%%%%%%%%%%%%%%%%%%%%%%%%%%%%%%%%%%%%%%%%%%%%%%%%
\newpage
\setcounter{secnumdepth}{0}
\section{Fundamentos Teóricos}

%\epigrafe{Everyone in this country should learn how to program a computer, because it teaches you how to think.}
%              {\textsc{Steve Jobs}}
              
El fundamento teórico de esta investigación se basa en los siguientes puntos clave:

1. Redes Neuronales: Se analizarán las redes neuronales como modelos computacionales inspirados en el funcionamiento del cerebro humano. Se abordarán sus componentes fundamentales, como las neuronas, las conexiones sinápticas y los algoritmos de aprendizaje. Se explorarán diferentes tipos de redes neuronales, como las redes neuronales convolucionales (CNN) y las redes neuronales recurrentes (RNN), y se discutirá su aplicación en el reconocimiento de sonido \cite{garciamurillo, frro,gtocoma,ehu}.

2. Reconocimiento de Sonido: Se examinarán los conceptos y métodos utilizados para el reconocimiento de sonido. Se analizarán técnicas de extracción de características de señales de audio, como el análisis espectral y el análisis temporal, así como algoritmos de clasificación y detección de patrones aplicados al reconocimiento de sonido. Se revisarán enfoques tradicionales, como el modelado oculto de Markov (HMM), y se enfatizará en el uso de redes neuronales para mejorar la precisión y eficiencia del reconocimiento de sonido \cite{tesisipn}.

3. Procesamiento de Señales de Audio: Se explorarán los fundamentos del procesamiento de señales de audio en el contexto del reconocimiento de sonido en la industria. Se abordarán técnicas de filtrado, separación de fuentes, cancelación de ruido y mejora de la calidad del sonido. Se discutirán algoritmos y enfoques específicos utilizados para resolver desafíos comunes en entornos industriales, como la presencia de ruido de fondo y la variabilidad de las condiciones de grabación \cite{cevallos}.

4. Los ventiladores son dispositivos que generan una corriente de aire para refrescar y ventilar el ambiente en el que se encuentran, y son importantes para mantener una temperatura adecuada y prolongar la vida útil de los componentes en las máquinas. En la industria, los ventiladores industriales son esenciales para mantener la calidad del aire y la temperatura en los procesos productivos, y pueden resistir condiciones operativas severas. 

5. Aplicaciones Industriales: Se analizará el papel de las redes neuronales en la industria, específicamente en la reducción de costos de producción, la optimización de procesos y la mejora de la calidad del producto. Se presentarán casos de estudio y ejemplos de implementaciones exitosas de redes neuronales en entornos industriales, destacando los beneficios obtenidos, como la eficiencia mejorada, la detección de defectos y la automatización de tareas de inspección \cite{garciamurillo, gtocoma,cevallos,ehu}.

6. Estado del Arte: Se realizará una revisión exhaustiva de la literatura científica y técnica más actualizada en el campo del reconocimiento de sonido basado en redes neuronales. Se identificarán las investigaciones y avances más relevantes, así como las tendencias emergentes y las áreas de investigación prometedoras. Se resumirán las principales contribuciones y se establecerá la base para el desarrollo de la investigación propuesta \cite{scribd}.
\newpage
\subsection{Problematica}

El reconocimiento de sonido es un campo de investigación relevante en la actualidad, con aplicaciones significativas en la industria y el procesamiento de señales de audio. Las redes neuronales han demostrado ser una herramienta eficaz para abordar diversos desafíos en estos ámbitos, como la reducción de costos de producción, la optimización de cadenas de suministro, la detección de defectos y la mejora de la calidad del producto en la industria de la maquinaria y fábricas. Asimismo, en el procesamiento de señales de audio, las redes neuronales son cruciales para mejorar la calidad del sonido, la separación de fuentes, la detección y eliminación de ruido, y la personalización del contenido auditivo.

Sin embargo, a pesar de los avances en el uso de redes neuronales en el reconocimiento de sonido, todavía existen desafíos y áreas de investigación que requieren atención. Además, se requiere investigar y desarrollar enfoques más eficientes y precisos para el procesamiento de señales de audio, con el fin de lograr una reproducción de sonido de alta calidad y una experiencia auditiva mejorada en los entornos industriales.

Por lo tanto, el presente estudio tiene como objetivo abordar estos desafíos y explorar las oportunidades de investigación en el campo del reconocimiento de sonido basado en redes neuronales, con un enfoque particular en la industria. Se busca identificar las áreas clave donde las redes neuronales pueden aportar mejoras significativas en términos de eficiencia, calidad del producto y experiencia auditiva.

En resumen, el problema planteado se centra en cómo aprovechar de manera óptima las redes neuronales en el reconocimiento de sonido para maximizar la eficiencia y calidad en la industria, abordando desafíos específicos en el procesamiento de señales de audio y su aplicación en entornos industriales mediante un prototipo embebido \cite{agudoestudio, atriaredes, garciamendoza, gomezarmenta, openwebinars, seorl2014audiologia}.
