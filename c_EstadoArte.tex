%!TEX root = Protocolo.tex
%%%%%%%%%%%%%%%%%%%%%%%%%%%%%%%%%%%%%%%%%%%%%%%%%%%%%%%%%%%%
%% Capítulo 2 / Fundamentos Teóricos y Problemática
%%%%%%%%%%%%%%%%%%%%%%%%%%%%%%%%%%%%%%%%%%%%%%%%%%%%%%%%%%%%
\newpage
\setcounter{secnumdepth}{0}
\section{Estado del arte}

%\epigrafe{Everyone in this country should learn how to program a computer, because it teaches you how to think.}
%              {\textsc{Steve Jobs}}
              
El reconocimiento de sonido es un campo de investigación relevante en la actualidad, con aplicaciones significativas tanto en la industria como en el procesamiento de señales de audio. Las redes neuronales desempeñan un papel importante en este campo, brindando soluciones efectivas y eficientes. En la industria de la maquinaria, las redes neuronales se utilizan para reducir costos de producción, optimizar cadenas de suministro, detectar defectos y mejorar la calidad del producto, lo que conduce a una mayor eficiencia y minimización de errores. Por otro lado, en el procesamiento de señales de audio, las redes neuronales son cruciales para mejorar la calidad del sonido, la separación de fuentes, la detección y eliminación de ruido, y la personalización del contenido auditivo, lo que contribuye directamente a la calidad de los productos y procesos en la industria. Este documento aborda la importancia y las aplicaciones de las redes neuronales en el reconocimiento de sonido implementando un prototipo embebido, proporcionando un panorama de los avances y las oportunidades de investigación en este campo \cite{agudoestudio, garciamendoza, gomezarmenta, leotronics, matich2018redes, atriaredes}.
