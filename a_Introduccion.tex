%!TEX root = Protocolo.tex
%%%%%%%%%%%%%%%%%%%%%%%%%%%%%%%%%%%%%%%%%%%%%%%%%%%%%%%%%%%%
%% Capítulo 1 / Introducción
%%%%%%%%%%%%%%%%%%%%%%%%%%%%%%%%%%%%%%%%%%%%%%%%%%%%%%%%%%%%

\newpage
\setcounter{secnumdepth}{0}
\addcontentsline{toc}{chapter}{Capítulo 1}
\section{Introducción}

El reconocimiento de sonido es un campo de investigación relevante en la actualidad, con aplicaciones significativas en la industria y el procesamiento de señales de audio. Las redes neuronales han demostrado ser una herramienta eficaz para abordar diversos desafíos en estos ámbitos, como la reducción de costos de producción, la optimización de cadenas de suministro, la detección de defectos y la mejora de la calidad del producto en la industria de la maquinaria y fábricas. Asimismo, en el procesamiento de señales de audio, las redes neuronales son cruciales para mejorar la calidad del sonido, la separación de fuentes, la detección y eliminación de ruido, y la personalización del contenido auditivo.

Sin embargo, a pesar de los avances en el uso de redes neuronales en el reconocimiento de sonido, todavía existen desafíos y áreas de investigación que requieren atención. Además, se requiere investigar y desarrollar enfoques más eficientes y precisos para el procesamiento de señales de audio, con el fin de lograr una reproducción de sonido de alta calidad y una experiencia auditiva mejorada en los entornos industriales.

Por lo tanto, el presente estudio tiene como objetivo abordar estos desafíos y explorar las oportunidades de investigación en el campo del reconocimiento de sonido basado en redes neuronales, con un enfoque particular en la industria. Se busca identificar las áreas clave donde las redes neuronales pueden aportar mejoras significativas en términos de eficiencia, calidad del producto y experiencia auditiva.

En resumen, el problema planteado se centra en cómo aprovechar de manera óptima las redes neuronales en el reconocimiento de sonido para maximizar la eficiencia y calidad en la industria, abordando desafíos específicos en el procesamiento de señales de audio y su aplicación en entornos industriales mediante un prototipo embebido \cite{agudoestudio, atriaredes, garciamendoza, gomezarmenta, openwebinars, seorl2014audiologia}.